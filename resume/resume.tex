%-------------------------
% Resume in Latex
% Author : Sourabh Bajaj
% License : MIT
%------------------------


\documentclass[a4paper, 10.5pt]{article}
\usepackage{graphicx}
\graphicspath{ {images/} }
\usepackage{latexsym}
\usepackage[empty]{fullpage}
\usepackage{titlesec}
\usepackage{marvosym}
\usepackage[usenames,dvipsnames]{color}
\usepackage{verbatim}
\usepackage{enumitem}
\usepackage[pdftex]{hyperref}
\usepackage{fancyhdr}


\pagestyle{fancy}
\fancyhf{} % clear all header and footer fields
\fancyfoot{}
\renewcommand{\headrulewidth}{0pt}
\renewcommand{\footrulewidth}{0pt}

% Adjust margins
\addtolength{\oddsidemargin}{-0.475in}
\addtolength{\evensidemargin}{-0.475in}
\addtolength{\textwidth}{1in}
\addtolength{\topmargin}{-.5in}
\addtolength{\textheight}{1.0in}

\urlstyle{same}

\raggedbottom
\raggedright
\setlength{\tabcolsep}{0in}

% Sections formatting
\titleformat{\section}{
  \vspace{-6pt}\scshape\raggedright\large
}{}{0em}{}[\color{black}\titlerule \vspace{-6pt}]

%-------------------------
% Custom commands
\newcommand{\resumeItem}[2]{
  \item\small{
    \textbf{#1}{: #2 \vspace{-2pt}}
  }
}

\newcommand{\resumeItemV}[2]{
  \item\small{
    \textbf{#1}{#2 \vspace{-2pt}}
  }
}

\newcommand{\resumeSubheading}[4]{
  \vspace{-1pt}\item
    \begin{tabular*}{0.97\textwidth}{l@{\extracolsep{\fill}}r}
      \textbf{#1} & #2 \\
      \textit{\small#3} & \textit{\small #4} \\
    \end{tabular*}\vspace{-5pt}
}

\newcommand{\resumeSubheadingTwo}[4]{
  \vspace{-1pt}\item
    \begin{tabular*}{0.97\textwidth}{l@{\extracolsep{\fill}}r}
      \textbf{#1} \textit{\small (#3)} & \textit{\small #4} \\
      %\textit{\small#3} & \textit{\small #4} \\
    \end{tabular*}\vspace{-10pt}
}

\newcommand{\resumeSubItem}[2]{\resumeItem{#1}{#2}\vspace{-4pt}}

\renewcommand{\labelitemii}{$\circ$}

\newcommand{\resumeSubHeadingListStart}{\begin{itemize}[leftmargin=*]}
\newcommand{\resumeSubHeadingListEnd}{\end{itemize}}
\newcommand{\resumeItemListStart}{\begin{itemize}}
\newcommand{\resumeItemListEnd}{\end{itemize}\vspace{-5pt}}

%-------------------------------------------
%%%%%%  CV STARTS HERE  %%%%%%%%%%%%%%%%%%%%%%%%%%%%


\begin{document}

%----------HEADING-----------------
\begin{tabular*}{\textwidth}{l@{\extracolsep{\fill}}r}
  \textbf{\href{http://bhushansonawane.com/}{\Large Bhushan B. Sonawane}} & Email: \href{mailto:bhushansonawane94@gmail.com}{bhushansonawane94@gmail.com}\\
  \href{https://github.com/bhushan23}{github.com/bhushan23} & Mobile: +1 (631) 590 9644 \\
\end{tabular*}



%-----------EDUCATION-----------------
\section{Education}
  \resumeSubHeadingListStart
    \resumeSubheading
      {SUNY StonyBrook University}
      {StonyBrook, NY}
      {Master of Science in Computer Science (Data Science Specialization); GPA: 3.57$/4$}{Aug 2017 - May 2019}
      \resumeItemListStart
         \resumeItem{Thesis} {Solving Lighting Estimation problem using deep learning; Advisor: \href{http://www3.cs.stonybrook.edu/~samaras/}{Professor Dimitris Samaras};} 
        %\resumeItem{Research Lab} {\href{http://www3.cs.stonybrook.edu/~cvl/index.html}{Computer Vision Lab}}{}
        \resumeItem{Courses} {\href{http://www3.cs.stonybrook.edu/~minhhoai/courses/cse512/index.html}{Machine Learning}, \href{http://francesco.orabona.com/teaching.html}{Convex Optimization}, 
        Introduction to Computer Vision, Natural Language Processing, 
        \href{http://www3.cs.stonybrook.edu/~anshul/courses/cse544_s18/}{Probability and Statistics}, \href{http://www3.cs.stonybrook.edu/~cse537/index.html}{Artificial Intelligence}
        %\href{http://www3.cs.stonybrook.edu/~rezaul/CSE548-F17.html}{Analysis of Algorithm}
        }
        \resumeItem{Senior Research Assistant}{Converting high-resolution medical images into tiled-tiff format [C]} 
%        \resumeItemV{Member of Computer Vision Lab}
        \resumeItemListEnd
    \resumeSubheading
      {Vishwakarma Institute of Technology}{Pune, India}
      {Bachelor of Technology in Computer Engineering; GPA: 9.27$/$10}{Aug 2011 - May 2015}
  \resumeSubHeadingListEnd
%-----------PROJECTS-----------------
\section{Projects}
  \resumeSubHeadingListStart
  %  \resumeSubItem{\href{https://github.com/bhushan23/Light-Estimation/}{Face Illumination estimation}}

    \resumeSubItem{\href{https://github.com/bhushan23/GAN/tree/master/Co-Operative-GAN}{Co-Operative GANs}}{Training GANs is tricky and often leads to mode collapsing. Training multiple generators and copy weights of best performing to other generators at the end of epoch. All generators starts from the best spot on every epoch. This solves mode collapsing, saddle point and local minima problem in training;\textbf{ \href{https://github.com/bhushan23/GAN/tree/master/Co-Operative-GAN}{Source \& Results}}; [Python, PyTorch]}
    \resumeSubItem{\href{https://github.com/bhushan23/pytorch/blob/1303c014dc3580654173c43fc6cf3409e4ef0438/torch/optim/admm.py}{ADMM Optimizer in PyTorch}}{Implemented Alternating Direction Method of Multipliers(ADMM) optimizer for Lasso and Ridge regression in PyTorch. Tested on Diabetes dataset; Speed up of 1.6x compare to Scikit-Learn's state of the art Lasso and Ridge solver; \textbf{\href{https://github.com/bhushan23/ADMM/blob/master/REPORT_ADMM_IN_PYTORCH.pdf}{Report}, \href{https://github.com/bhushan23/pytorch/blob/1303c014dc3580654173c43fc6cf3409e4ef0438/torch/optim/admm.py}{Source},  \href{https://github.com/bhushan23/ADMM}{Results}}; [Python, PyTorch]}
%    \resumeSubItem{\href{https://github.com/bhushan23/Computer-Vision}{Deep Learning for Computer Vision}}{Implemented CNNs for CIFAR-10 object recognition; \href{https://github.com/bhushan23/GAN/}{Implemented Auencoders and GANs for image generation}; [Python, PyTorch]}
    \resumeSubItem{\href{https://github.com/bhushan23/SBU-ML-Assignment}{ML Algorithms}}{Implemented Ridge Regression, Lasso Solver, Support Vector Machine using Stochastic Gradient Descent and Quadratic Programming; Human Action recognition using CNN and RNN \textbf{\href{https://github.com/bhushan23/SBU-ML-Assignment}{Source}}; [Python, Matlab]}
   % \resumeSubItem{\href{https://github.com/bhushan23/SmartOff}{SmartOFF - Managing power supply of appliances for energy conservation}} {Home appliances consumes significant power in stand by mode; Internet of Things and Machine Learning solution; LSTM model for predicting appliances' usage pattern and control power supply accordingly. \textbf{Links:}\href{https://github.com/bhushan23/SmartOff/tree/master/Reports}{ Reports}, \href{https://github.com/bhushan23/SmartOff}{Source}; [Python, Scikit-learn, Keras]}
    \resumeSubItem{\href{https://github.com/bhushan23/SmartOff}{SmartOFF - Automate power supply of home appliances}} {IoT and ML solution; LSTM model for predicting appliances' usage pattern and predict when appliance will not be used and can be turned off. Used ESP8266 Microcontroller for communication. Client-Server model where Server devices using trained LSTM model sends signal to toggle power of respective device. \textbf{\href{https://github.com/bhushan23/SmartOff/tree/master/Reports}{Report}, \href{https://github.com/bhushan23/SmartOff}{Source}}; [Python, Scikit-learn, Keras]}
   % \resumeSubItem{Competitions}{Worked on \href{http://ai.bu.edu/visda-2018/}{Visual Domain Adaptation}, \href{https://www.crowdai.org/challenges/nips-2018-ai-for-prosthetics-challenge}{NIPS 2018: AI for Prosthetics}}
    % \resumeSubItem{\href{http://urc.marssociety.org/home}{University Rover Challenge}}{Member of StonyBrook Mars Rover Challenge Computer Vision team; Working on to create depth map for avoiding large obstacles using ZED camera; [OpenCV, Python, OnGoing]}
    %\resumeSubItem{\href{https://github.com/bhushan23/pytorch-rl}{PyTorch-RL}}{Deep Reinforcement Learning algorithms implementation ready to be used for PyTorch [Python, PyTorch, OnGoing]}
    \resumeSubItem{\href{https://www.crowdai.org/challenges/nips-2018-ai-for-prosthetics-challenge}{NIPS 2018 AI for Prosthetics Challenge}}{Using Reinforcement learning to model human with a prosthetic leg to walk and run; Using Deep Deterministic Policy Gradient and Proximal Policy Optimization
    [Python, Keras, PyTorch, Ongoing]}
    %Using \href{https://github.com/keras-rl/keras-rl}{keras-rl} library; [Python, Keras, PyTorch, Ongoing]} 
%    \resumeSubItem{\href{http://ai.bu.edu/visda-2018/}{Visual Domain Adaptation Challenge}}{Developing model to adapt between synthetic and real objects for detection; Developing a method of unsupervised domain adaptation for object classification with additional unknown categories; [Python, PyTorch, OnGoing]}
    \resumeItemListEnd
%\resumeSubItem{GroupPlay}
%      {Synchronize all devices for audio playback over wifi. [Android, Java]}
%      \resumeItem{Ani Malware Software}{Detecting malicious and %duplicate files using MD5 and SHA-256 algorithm; \\Used Avast %antivirus virus dataset. [Java]}
  \resumeSubHeadingListEnd

%-----------Open Source--------------
\section{Open-Source}
\resumeSubHeadingListStart
\resumeSubItem{\href{https://github.com/pytorch/pytorch}{PyTorch}}{\href{https://github.com/pytorch/pytorch/issues/9132}{torch.isInf, isFinite}; \href{https://github.com/pytorch/pytorch/issues/9546}{Negative indices with torch.narrow}; \href{https://github.
com/pytorch/pytorch/issues/9530}{Keys from load state};\href{https://docs.google.com/spreadsheets/d/1oFntDigXwlkAzJH_T6XagQD6ia-9H4YmqZMTWmPbz8w/edit?usp=sharing}{ \textbf{Status}} [Python, C++]}
%\resumeSubItem{\href{https://github.com/bhushan23/LogicalVision2}{Logical Vision}} {Polygon Detection; Implemented KNN using ml-pack. [Prolog, C++, Python, OpenCV]}
\resumeItemListEnd


%-----------EXPERIENCE-----------------
\section{Experience}
  \resumeSubHeadingListStart
    %\resumeSubheadingTwo
     % {Author - 'PyTorch Deep Learning Projects'}{Remote}
     % {\href{https://www.packtpub.com/}{Packt Publications}}{Aug 2018 - Current}
     % \resumeItemListStart
     %   \resumeSubItem{Book}{Writing book for Students, Professionals and Researchers getting started in Deep Learning}
     %   \resumeSubItem{Areas being covered}{Recommendation Systems, Computer Vision, Natural Language Processing, Reinforcement Learning, Generative Adversarial Networks, Self Driving Car}
    % \resumeItemListEnd
    \resumeSubheadingTwo
      {Computer Vision Lab, Stony Brook University}{StonyBrook, NY}
      {Master's Thesis}{Jan 2017 - Current}
      \resumeItemListStart
        \resumeSubItem{Face Illumination Estimation}{Estimating face illumination using spherical harmonics; Used domain adapation to overcome lack of ground truth. Used GANs for domain adaptation to re-use network trained on synthetic data.Used \href{https://drive.google.com/file/d/1RvyCiDMg--jyO8lLBvopp0o271LvREoa/view}{SIRFS} method for generating shading, albedo, normal and lighting for synthetic and CelebA dataset. Enhanced \href{https://jonbarron.info/}{Jon Barron's} \href{https://github.com/bhushan23/SIRFS}{SIRFS \_fast implementation}; \textbf{\href{https://github.com/bhushan23/Light-Estimation/blob/master/Lighting_Estimation_Report.pdf}{Report}, \href{https://github.com/bhushan23/Light-Estimation/}{Source \& Results}}; [Python, Matlab, PyTorch]}
        \resumeSubItem{Modeling Illumination in Neural Network}{Estimating lighting sources, direction is very important for face editing tasks. Ongoing research. [Ongoing, Python, Matlab, PyTorch, CVPR 2019]}
      \resumeItemListEnd
    \resumeSubheadingTwo
      {Nvidia}{Santa Clara, CA}
      {Intern, SPIR-V Compiler}{May 2018 - Aug 2018}
      \resumeItemListStart
      \resumeItem{Confidential}{In the intersection of LLVM compiler and Machine Learning [C++, LLVM, Python]}
      %  \resumeSubItem{Knobs Infrastructure} {Knobs infrastructure to allow compiler debugging and experimentation [C++, LLVM]}
       % \resumeSubItem{Phase Dispatcher} {Compiler phase ordering and parameter tuning framework to enable compile time and run time performance exploration for compiler. [C++, LLVM]}
      \resumeItemListEnd
    \resumeSubheadingTwo
      {Nvidia}{Pune, India}
      {System Software Engineer, Optimizing Compiler}{Jun 2015 - Jul 2017}
      \resumeItemListStart
        \resumeSubItem{Optimizing compiler} {Worked on Nvidia Tegra graphics and CUDA compute compiler; Improved peephole optimizations; OpenGL/DX driver interfaces; Optimization for compile time improvements; Developed Profiling infrastructure; Worked on Tegra(Android, Nintendo) compiler issues; Worked on Coverity, GCov; [C/C++]}
        \resumeSubItem{Assembler}{Implemented DWARF 2.0 compliant debug frame support for CUDA 9.0. [C]}
      \resumeItemListEnd

    \resumeSubheadingTwo
      {Nvidia}{Pune, India}
      {Intern, Optimizing Compiler}{Jun 2014 - Apr 2015}
      \resumeItemListStart
        \resumeSubItem{\href{http://slides.com/bhushansonawane/deck/}{PBQP based Register Allocator}}{Implemented Partitioned Boolean Quadratic Problem based register allocator for Nvidia compiler; 98\% of existing tests improved (graphics and compute tests); [C++]}
      \resumeItemListEnd
     \resumeSubheadingTwo
      {Startup}{Pune, India}
      {Technology and Management Role}{Jan 2014 - Mar 2015}
      \resumeItemListStart
      \resumeSubItem{\href{http://metromidnight.com/}{MetroMidnight}}{Food delivery startup,\textbf{\href{https://www.quodeit.com/}{ QuodeIT:}}{ Programming screening platform}}
      \resumeItemListEnd
      %\resumeSubItem{SUNY Research Foundation}{Image parser for converting proprietary bio-medical images into tiled-tiff [C]}
    %  \resumeSubItem{{Vishwakarama Institute of Technology}} {Instructed undergraduate course     \href{https://www.hackerrank.com/coding-puzzles/}{'Problem Solving and Programming'}}
      
   \resumeSubHeadingListEnd



%\section{Other Experience}
%\resumeItemListStart
%\resumeSubItem{SUNY Research Foundation}{Image parser for converting proprietary bio-medical images into tiled-tiff [C]}
%\resumeSubItem{{Vishwakarama Institute of Technology}} {Instructed undergraduate course %\href{https://www.hackerrank.com/coding-puzzles/}{'Problem Solving and Programming'}}
%\resumeItemListEnd

%
%--------PROGRAMMING SKILLS------------
\section{Skills}
  \resumeSubHeadingListStart
    \item{
      \textbf{}{C++, C, Python, Java, PyTorch, Keras, Tensorflow, LLVM, Django, Grails, Android}
      \hfill
    }
  \resumeSubHeadingListEnd


%
%--------AWARDS------------
\section{Awards}
\resumeSubHeadingListStart
    \resumeSubItem{Project rank 1/126}{PBQP based register allocator project secured first place at VIT(2015)}
    \resumeSubItem{Paper Presentation rank 2/88}{Page Replacement algorithm using hashing got second place at Papyrus, VIT(2014)}
      \resumeSubItem{Competitive Programming}{\textbf{Rank 2/66} in \href{https://www.kaggle.com/c/hw2-activity-recognition-cse512-spr18/leaderboard}{Kaggle Competition} for Human Acticity Recognition(2018); \textbf{Rank 1/600} at programming contest(C-Athlon)(2014); Qualified for \textbf{ACM ICPC} Amritapuri regionals(2013)}
   \resumeSubHeadingListEnd
%-------------------------------------------

\end{document}

